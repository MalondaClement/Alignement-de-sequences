\paragraph{Question 1}
Soient $(\bar x,\bar y)$ et $(\bar u, \bar v)$ deux alignements respectivement de $(x, y)$ et $(u, v)$. $\bar x$ et $\bar y$ sont aligné et donc sont de même longueur $(|\bar x| = |\bar y|)$. De même pour l'alignement $(\bar u, \bar v)$, $(|\bar x| = |\bar y|)$.\\
A partir de ces deux affirmations, nous pouvont dire que la concaténation de $\bar x$ et $\bar u$, $\bar x.\bar u$ ainsi que la concaténation de $\bar y$ et $\bar v$, $\bar y.\bar v$ sont de même longueur.

\paragraph{Question 2}
Soient $x \in \sum^*$ un mot de longueur $n$ et $y \in \sum^*$ un mot de longueur $m$, la longueur maximale d'un alignement de $(x,y)$ est $n+m-1$.\\
On prend par exemple $x=ATCG$ et $y=GCTGA$, l'aligement de longueur maximale est :

$$\bar x : ATCG\_\_\_\_$$
$$\bar y : \_\_\_GCTGA$$
