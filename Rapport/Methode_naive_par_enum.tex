\paragraph{Question 3}.\\
Soit $x \in \sum$ un mot de longueur n, on souhaite ajouter exactement $k$ gaps dans ce mot pour obtenir le mot $\bar x$.\\
Il existe, sur un ensemble de $n + k$ valeurs, $n^k+1$ combinaisons possibles.
\paragraph{Question 4}.\\
$|\bar x| = n \leq m = |y|$ où $\bar x$ est le mot $x$ avec $k$ gaps.\\
On a donc $|\bar x| - |y| = k_y$ le nombre de gaps que l'on doit insérer dans le mot $y$.
\paragraph{Question 5}.\\

\paragraph{Question 6}.\\
La complexité spatiale qu'aurait un algorithme naïf qui énumère tous les alignements de deux mots serait en $O$ de la taille du plus grand alignement multiplié par la nombre d'alignements.
\paragraph{Tâche A}
Il est possible de résoudre les instances fournies en moins d'une minute pour celles qui sont de taille 10 et 12.\\
La consommation mémoire nécessaire au fonctionnement de cette méthode est d'environ 4900Ko.
Nous utiliserons la commande suivant pour utiliser le programme :
\begin{lstlisting}
time python3 TacheA.py un_instance.adn
\end{lstlisting}

