\paragraph{Question 15}.\\
On a $i > 0$ et $D(i,j) = D(i-1,j)+c_{del}$\\
Si $\exists(\bar s, \bar t) \in Al^*(i-1,j)+c_{del}$\\
Montrons alors que $(\bar s.x_i)\in Al^*(i,j)$\\
$Al^*(i-1,j)$ veut dire qu'il existe un alignement $(\bar s, \bar t)$ de $(x_[1...i-1], y_[1...j])$ tel que $C(\bar s, \bar t) = D(i-1, j)$. Par définition $c_{del}$ est le coût d'une supression qui consiste à encoder un gap dans $\bar y$ pour marquer la supression de la lettre de $x$ qui est parallèle à ce gap dans $\bar y$.
Par définition on va donc ajouter un élément dans $\bar y$ sans avancer dans le mot $y$ et on parallèle on avance d'une lettre dans x.\\
On aura donc $D(i-1,j)+c_{del} = D(i,j)$\\
Donc il existera un alignement $(\bar u, \bar v)$ tel que $C(\bar u, \bar v) = D(i,j)$ avec $\bar u = \bar s.x_i$ et $\bar v = \bar t.-$\\
Alors on a bien $(\bar s.x, \bar t.-) \in Al^*(i,j)$.
\paragraph{Question 16}.\\
\begin{lstlisting}
SOL_1 (x, y, T) :
    u <- ""
    v <- ""
    n <- x.taille
    m <- y.taille
    i <- n
    j <- m

    si (n==0) alors :
        u <- m*"-"
        v <- y
        return (u,v)
    sinon si (m==0) alors :
        u <- x
        v <- n*"-"
        return (u,v)

    tant que (i >= 1) or (j >= 1) faire :

        si (i == 1) alors :
            u <- x[i-1] + u
            si (j==1) faire:
                v <- y[j-1] + v
                return (u,v)
            sinon :
                tant que (j>=1) faire :
                    u <- "-" + u
                    v <- y[j-1] + v
                    j <- j -1
                v <- y[j-1] + v
                retourne (u, v)

        sinon si j == 1 alors :
            v <- y[j-1] + v
            tant que (i>1) alors :
                v <- "-" + v
                u <- x[i-1] + u
                i <- i -1
            u <- u[i-1] + u
            retourne (u, v)

        sinon si (T[i][j]) == (T[i-1][j] + 2) alors :
            u <- x[i-1] + u
            v <- "-" + v
            i <- i - 1

        sinon si (T[i][j]) == (T[i][j-1] + 2) alors :
            u <- "-" + u
            v <- y[j-1] + v
            j <- j - 1

        sinon si T[i][j] == (T[i-1][j-1] + cout.c_sub (x[i-1], yfaire [j-1])) :
            u <- x[i-1] + u
            v <- y[j-1] + v
            i <- i - 1
            j <- j - 1
    retourne (u, v)
\end{lstlisting}
\paragraph{Question 17}.\\
La résolution du problème ALI, par combinaison des algorithmes DIST\_1 et SOL\_1, se fait avec un complexité en $O(n\times m)$ car chacun des deux algorithmes et que SOL\_1 fait appelle une fois à DIST\_1.
\paragraph{Question 18}.\\
La complexité spatiale de DIST\_1 est en $\Theta(n\times m)$ et celle de SOL\_1 est en $O(n+m)$.
\newpage
\paragraph{Tâche B}.\\
Nous utiliserons la commande suivant pour utiliser le programme :
\begin{lstlisting}
time python3 sol_1.py un_instance.adn
\end{lstlisting}
\img{Images/mesureB.png}{Courbe de consommation de temps (en s) CPU en fonction de la taille |x|}

