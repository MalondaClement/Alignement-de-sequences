On considère $(x,y)\in \sum^* \times \sum^*$ un couple de mots de longueurs respectives $n$ et $m$.
\paragraph{Question 7}.\\
On distingue trois cas possible :\begin{itemize}
\item[-] si $\bar{u}_j =$ - alors $\bar{v_l} = y_j$ car par définition il n'est pas possible d'avoir deux gaps face à face dans un alignement.
\item[-] si $\bar{v_l} =$ - alors $\bar{u_l} = x_i$ car par définition il n'est pas possible d'avoir deux gaps face à face dans un alignement.
\item[-] si $\bar{u_l} \neq$ - et $\bar{v_l} \neq$ - alors $\bar{v_l} = \bar{u_l}$ car le seul moyen de ne pas avoir de gap en une position $l$ de la liste est que les lettres des deux mots à cet endroit soient les mêmes.
\end{itemize}
\paragraph{Question 8}
$$C(\bar u, \bar v) = C(\bar u_{[1...l-1]}, \bar v_{[1...l-1]}) + \left \{
    \begin{array}{rcl}
        c_{ins} \text{ si } \bar u_l = -\\
        c_{del} \text{ si } \bar v_l = -\\
        c_{sub} \text{ sinon }
    \end{array}
\right.$$
\paragraph{Question 9}.\\
Soit $(\bar u, \bar v)$ un alignement de $(x_{[1...i]} ,y_{[1...j]})$\\
On sait que
$$
 \begin{array}{r c l}
      D(i,j) & = & d(x_{[1...i]} ,y_{[1...j]}) \\
      D(i,j) & = & min\{\bar u, \bar v\}\\
   \end{array}
$$
\subparagraph*{Dans le cas n\deg1 :}on a $\bar v_l = y_j$ et $\bar u_l = -$, ce qui une insertion. Puis, pour calculer la distance des éléments précédents, nous allons devoir prendre l'élément précédent dans $y$ mais pas dans $x$ car la place $x_i$ est occupé par un gap.
\subparagraph*{Dans le cas n\deg2 :}on a $\bar u_l = x_i$ et $\bar v_l = -$, ce qui une supression. Puis, pour calculer la distance des éléments précédents, nous allons devoir prendre l'élément précédent dans $x$ mais pas dans $y$ car la place $y_j$ est occupé par un gap.
\subparagraph*{Dans le cas n\deg3 :}il n'y a aucun gap donc on prendra dans les deux mots les lettres précédentes $x_{[1...i-1]}$ et $y_{[1...j-1]}$.

\paragraph{Question 10}.\\
 $D(0, 0) = 0$ car les lettres d'un mot sont numérotées de 1 à $n$ avec $n$ la taille du mot, la position $(0,0)$ correspond au mot vide.
\paragraph{Question 11}.\\
$D(0, j) = m\times c_{ins} $ l'alignement du mot vide $w$ avec un mot de taille $m$ se fait par l'ajout de $m$ gaps dans $w$ face à chaque lettre\\
$D(i, 0) = n\times c_{dsl} $ l'alignement du mot vide $w$ avec un mot de taille $n$ se fait par l'ajout de $n$ gaps dans $w$ face à chaque lettre
\paragraph{Question 12}.\\
\begin{lstlisting}
DIST_1 (x, y) :
    n <- x.taille + 1
    m <- y.taille + 1
    T <- entier[n][m]

    pour i allant de 0 a n faire :
        pour j allant de 0 a m faire  :
            si i==0 faire :
                T[i][j] <- j*2

            sinon si j==0 faire:
                T[i][j] <- i*2
            sinon :
                T[i][j] <- min (T[i-1][j]+2, T[i][j-1] + 2, T[i-1][j-1] + \\
                c_sub (x[i-1], y[j-1]))

    return (T[n-1][m-1], T)
\end{lstlisting}
Nous retournons le couple constitué de la distance d'édition et du tableau de toutes les valeurs de $D$ car nous avons besoin de ce tableau dans SOL\_1.
\paragraph{Question 13}.\\
L'algorithme DIST\_1 utilise une matrice de taille $n \times m$, sa complexité spatiale est en $\Theta(n\times m)$.
\paragraph{Question 14}.\\
L'algorithme DIST\_1 est constitué de deux boucles imbriquées. La boucle intérieure ne fait que des opérations élémentaires en $O(1)$ donc la complexité temporelle est en $O(n\times m)$.
