
\paragraph{Question 21}
Pseudo-code de la fonction mot\_gaps :
\begin{lstlisting}
mot_gaps(k):
    res <- ""
    pour i allant de 0 a k-1 faire:
        res <- res + "-"
    retourne res
\end{lstlisting}
\paragraph{Question 22}
Pseudo-code de la fonction align\_lettre\_mot :
\begin{lstlisting}
align_lettre_mot(lettre, mot):
    i <- 0
    tant que lettre != mot[i] et i < mot.taille:
        i <- i + 1
    si i < mot.taille alors:
        x <- mot_gaps(i) + lettre + mot_gaps(mot.taille - i - 1)
    sinon:
        mot <- mot + "-"
        lettre <- mot_gaps(mot.taille) + lettre
    retourne(lettre,mot)

\end{lstlisting}

\paragraph{Question 23}
\paragraph{Question 24}
Pseudo-code de la fonction SOL\_2 en considérant que l'on possède une fonction coupure
\begin{lstlisting}
SOL_2(x, y):
    si y.taille == 0 alors:
        retourne (x,mot_gaps(x.taille))
    si x.taille == 1 et si y.taille == 1 alors:
        si x == y alors :
            retourne (x,y)
        sinon retourne (x+"-", "-"+y)
    sinon:
        res1 <- SOL_2(x[:x.taille/2],y[:coupure(x,y)])
        res2 <- SOL_2(x[x.taille/2:],y[coupure(x,y:)])
        retourne (res1[0]+res2[0], res1[1]+res2[1])
\end{lstlisting}
\paragraph{Question 25}
Pseudo-code de la fonction coupure :
\begin{lstlisting}
coupure (x, y, T) :
    n <- len(x)+1
    m <- len(y)+1
    p <- (n-1)//2
    I <- np.zeros((n, m))
    i <- 1
    j <- 1

    pour i allant de 1 a n faire:
        pour j allant de 1 a  m faire :

            si (i<p) faire :
                I[i][j] <- 0

            sinon si (i==p) faire :
                I[i][j] <- j

            sinon :
                s <- min (T[i-1][j], T[i][j-1], T[i-1][j-1])

                si (s == T[i-1][j]) faire :
                    I[i][j] <- I[i-1][j]

                sinon si (s == T[i][j-1]) faire :
                    I[i][j] <- I[i][j-1]

                sinon si (s == T[i-1][j-1]) faire :
                    I[i][j] <- I[i-1][j-1]
    return (I[n-1][m-1])
\end{lstlisting}
\paragraph{Question 26}
\paragraph{Question 27}
\paragraph{Question 28}
\paragraph{Tâche D}
\paragraph{Question 29}
