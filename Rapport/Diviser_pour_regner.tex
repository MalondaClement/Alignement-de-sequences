
\paragraph{Question 21}
Pseudo-code de la fonction mot\_gaps :
\begin{lstlisting}
mot_gaps(k):
    res <- ""
    pour i allant de 0 a k-1 faire:
        res <- res + "-"
    retourne res
\end{lstlisting}
\paragraph{Question 22}
Pseudo-code de la fonction align\_lettre\_mot :
\begin{lstlisting}
align_lettre_mot(lettre, mot):
    i <- 0
    tant que lettre != mot[i] et i < mot.taille:
        i <- i + 1
    si i < mot.taille alors:
        x <- mot_gaps(i) + lettre + mot_gaps(mot.taille - i - 1)
    sinon:
        mot <- mot + "-"
        lettre <- mot_gaps(mot.taille) + lettre
    retourne(lettre,mot)

\end{lstlisting}

\paragraph{Question 23}.\\
On a les coûts suivants :
$$c_{ins} = c_{del} = 3 \text{ et } c_{sub} = \left \{
    \begin{array}{l}
        0 \text{ si } a = b\\
        5 \text{ si } a\text{ et }b \text{ sont deux consonnes/voyelles distinctes}\\
        7 \text{ sinon }
    \end{array}
\right.$$
On a : $x = $ BALLON et $y = $ ROND donc $x^1 = $ BAL, $x^2 = LON$, $y^1 = $ RO et $y^2 = $ ND.
$$\bar s : B\_A\_L$$
$$\bar t : \_R\_O\_$$
$$\bar u : LON\_$$
$$\bar v : \_\_ND$$
Les coûts de ces deux alignements sont respectivement de $13$ et $9$, ce qui fait un coût total de $22$.\\
Or il est possible de faire un autre alignement :
$$\bar s : \_BALLON\_$$
$$\bar t : R\_\_\_OND$$
Le coût de l'alignement si dessus est de $5\times 3 + 5 = 20$ soit un coût inférieur à celui réalisé avec la coupure.

\paragraph{Question 24}
Pseudo-code de la fonction SOL\_2 en considérant que l'on possède une fonction coupure
\begin{lstlisting}
SOL_2(x, y):
    si y.taille == 0 alors:
        retourne (x,mot_gaps(x.taille))
    si x.taille == 1 et si y.taille == 1 alors:
        si x == y alors :
            retourne (x,y)
        sinon retourne (x+"-", "-"+y)
    sinon:
        res1 <- SOL_2(x[:x.taille/2],y[:coupure(x,y)])
        res2 <- SOL_2(x[x.taille/2:],y[coupure(x,y:)])
        retourne (res1[0]+res2[0], res1[1]+res2[1])
\end{lstlisting}
\paragraph{Question 25}
Pseudo-code de la fonction coupure :
\begin{lstlisting}
coupure (x, y, T) :
    n <- len(x)+1
    m <- len(y)+1
    p <- (n-1)//2
    I <- np.zeros((n, m))
    i <- 1
    j <- 1

    pour i allant de 1 a n faire:
        pour j allant de 1 a  m faire :

            si (i<p) faire :
                I[i][j] <- 0

            sinon si (i==p) faire :
                I[i][j] <- j

            sinon :
                s <- min (T[i-1][j], T[i][j-1], T[i-1][j-1])

                si (s == T[i-1][j]) faire :
                    I[i][j] <- I[i-1][j]

                sinon si (s == T[i][j-1]) faire :
                    I[i][j] <- I[i][j-1]

                sinon si (s == T[i-1][j-1]) faire :
                    I[i][j] <- I[i-1][j-1]
    return (I[n-1][m-1])
\end{lstlisting}
\paragraph{Question 26}.\\
La complexité spatiale de la fonction coupure est $\Theta(m)$ car elle utilise un tableau de taille $2\times m$ pour stocker les valeurs $D(i,j)$ données par l'équation de réccurrence et que l'on utilise uniquement un tableau à deux lignes car se sont les seuls utiles au moment du calcul.
\paragraph{Question 27}.\\
La complexité spatiale de la fonction SOL\_2 est en $O((n+m)^2)$ car elle utilise deux chaînes de caractères pour stocker solution est utilise des appels récursifs.
\paragraph{Question 28}.\\
La complexité temporelle de la fonction coupure est en $O(n\times m)$ car elle parcourt un tableau de taille $2 \times m$ mais celui-ci doit être mis-à-jour et donc parcouru à nouveau jusqu'à un maximum de $n$ fois.
\paragraph{Tâche D}.\\
Nous utiliserons la commande suivant pour utiliser le programme :
\begin{lstlisting}
time python3 sol_2.py un_instance.adn
\end{lstlisting}
\paragraph{Question 29}
