\documentclass[12pt]{article}
\usepackage[utf8]{inputenc}
\usepackage[T1]{fontenc}
\usepackage[french]{babel}
\usepackage{fullpage}
\usepackage{color}
\usepackage[table]{xcolor}
\usepackage{listings}

\def\deg{\ensuremath{^\circ}}% symbole °

\begin{document}
\title{Mini-Projet : Alignement de séquences \\ LU3IN003 - Sorbonne Université}
\author{Amann Emmanuelle \& Malonda Clément}
%\date{}
\maketitle
\tableofcontents
\newpage

\section{Introduction}
\paragraph{Question 1}.\\
Soient $(\bar x,\bar y)$ et $(\bar u, \bar v)$ deux alignements respectivement de $(x, y)$ et $(u, v)$. $\bar x$ et $\bar y$ sont aligné et donc sont de même longueur $(|\bar x| = |\bar y|)$. De même pour l'alignement $(\bar u, \bar v)$, $(|\bar x| = |\bar y|)$.\\
A partir de ces deux affirmations, nous pouvont dire que la concaténation de $\bar x$ et $\bar u$, $\bar x.\bar u$ ainsi que la concaténation de $\bar y$ et $\bar v$, $\bar y.\bar v$ sont de même longueur.

\paragraph{Question 2}.\\
Soient $x \in \sum^*$ un mot de longueur $n$ et $y \in \sum^*$ un mot de longueur $m$, la longueur maximale d'un alignement de $(x,y)$ est $n+m-1$.\\
On prend par exemple $x=ATCG$ et $y=GCTGA$, l'aligement de longueur maximale est :

$$\bar x : ATCG\_\_\_\_$$
$$\bar y : \_\_\_GCTGA$$

\section{Méthode naïve par énumération}
\paragraph{Question 3}
\paragraph{Question 4}
\paragraph{Question 5}
\paragraph{Question 6}
\paragraph{Tâche A}
Il est possible de résoudre les instances fournies en moins d'une minute pour celles qui sont de tailles 10 et 12.\\
La consommation mémoire nécessaire au fonctionnement de cette méthode est d'environ 4900Ko.


\newpage
\section{Programmation dynamique}
\subsection{pour le calcul de la distance d'édtion}
On considère $(x,y)\in \sum^* \times \sum^*$ un couple de mots de longueurs respectives $n$ et $m$.
\paragraph{Question 7}.\\
On distingue trois cas possible :\begin{itemize}
\item[-] si $\bar{u}_j =$ - alors $\bar{v_l} = y_j$ car par définition il n'est pas possible d'avoir deux gaps face à face dans un alignement.
\item[-] si $\bar{v_l} =$ - alors $\bar{u_l} = x_i$ car par définition il n'est pas possible d'avoir deux gaps face à face dans un alignement.
\item[-] si $\bar{u_l} \neq$ - et $\bar{v_l} \neq$ - alors $\bar{v_l} = \bar{u_l}$ car le seul moyen de ne pas avoir de gap en une position l de la liste est que les lettres des deux mots à cette endroit soient les mêmes.
\end{itemize}
\paragraph{Question 8}
$$C(\bar u, \bar v) = C(\bar u_{[1...l-1]}, \bar v_{[1...l-1]}) + \left \{
    \begin{array}{rcl}
        c_{ins} si \bar u_l = -\\
        c_{del} si \bar v_l = -\\
        c_{sub} sinon
    \end{array}
\right.$$
\paragraph{Question 9}.\\
Soit $(\bar u, \bar v)$ un alignement de $(x_{[1...i]} ,y_{[1...j]})$\\
On sait que
$$
 \begin{array}{r c l}
      D(i,j) & = & d(x_{[1...i]} ,y_{[1...j]}) \\
      D(i,j) & = & min\{\bar u, \bar v\}\\
   \end{array}
$$
\subparagraph*{Dans le cas n\deg1 :}on a $\bar v_l = y_j$ et $\bar u_l = -$, ce qui une insertion. Puis, pour calculer la distance des éléments précédents, nous allons devoir prendre l'élément précédent dans $y$ mais pas dans $x$ car la place $x_i$ est occupé par un gap.
\subparagraph*{Dans le cas n\deg2 :}on a $\bar u_l = x_i$ et $\bar v_l = -$, ce qui une supression. Puis, pour calculer la distance des éléments précédents, nous allons devoir prendre l'élément précédent dans $x$ mais pas dans $y$ car la place $y_j$ est occupé par un gap.
\subparagraph*{Dans le cas n\deg3 :}il n'y a aucun gap donc on prendra dans les deux mots les lettres précédentes $x_{[1...i-1]}$ et $y_{[1...j-1]}$.

\paragraph{Question 10}.\\
 $D(0, 0) = 0$ car les lettres d'un mot sont numérotées de 1 à $n$ avec $n$ la taille du mot, la position $(0,0)$ correspond au mot vide.
\paragraph{Question 11}.\\
$D(0, j) = m\times c_{ins} $ l'alignement du mot vide $w$ avec un mot de taille $m$ se fait par l'ajout de $m$ gaps dans $w$ face à chaque lettre\\
$D(i, 0) = n\times c_{dsl} $ l'alignement du mot vide $w$ avec un mot de taille $n$ se fait par l'ajout de $n$ gaps dans $w$ face à chaque lettre
\paragraph{Question 12}.\\
\begin{lstlisting}
DIST_1 (x, y) :
    n <- x.taille + 1
    m <- y.taille + 1
    T <- entier[n][m]

    pour i allant de 0 a n faire :
        pour j allant de 0 a m faire  :
            si i==0 faire :
                T[i][j] <- j*2

            sinon si j==0 faire:
                T[i][j] <- i*2
            sinon :
                T[i][j] <- min (T[i-1][j]+2, T[i][j-1] + 2, T[i-1][j-1] + \\
                c_sub (x[i-1], y[j-1]))

    return (T[n-1][m-1], T)
\end{lstlisting}
Nous retournons le couple constitué de la distance d'édition et du tableau de toutes les valeurs de $D$ car nous avons besoin de ce tableau dans SOL\_1.
\paragraph{Question 13}.\\
L'algorithme DIST\_1 utilise une matrice de taille $n \times m$, sa complexité spatiale est en $O(n\times m)$.
\paragraph{Question 14}.\\
L'algorithme DIST\_1 est constitué de deux boucles imbriquées. La boucle intérieure ne fait que des opérations élémentaires en $O(1)$ donc la complexité temporelle est en $O(n\times m)$.

\subsection{pour le calcul d'un alignement optimal}
\paragraph{Question 15}.\\
On a $i > 0$ et $D(i,j) = D(i-1,j)+c_{del}$\\
Si $\exists(\bar s, \bar t) \in Al^*(i-1,j)+c_{del}$\\
Montrons alors que $(\bar s.x_i)\in Al^*(i,j)$\\
$Al^*(i-1,j)$ veut dire qu'il existe un alignement $(\bar s, \bar t)$ de $(x_[1...i-1], y_[1...j])$ tel que $C(\bar s, \bar t) = D(i-1, j)$. Par définition $c_{del}$ est le coût d'une supression qui consiste à encoder un gap dans $\bar y$ pour marquer la supression de la lettre de $x$ qui est parallèle à ce gap dans $\bar y$.
Par définition on va donc ajouter un élément dans $\bar y$ sans avancer dans le mot $y$ et on parallèle on avance d'une lettre dans x.\\
On aura donc $D(i-1,j)+c_{del} = D(i,j)$\\
Donc il existera un alignement $(\bar u, \bar v)$ tel que $C(\bar u, \bar v) = D(i,j)$ avec $\bar u = \bar s.x_i$ et $\bar v = \bar t.-$\\
Alors on a bien $(\bar s.x, \bar t.-) \in Al^*(i,j)$.
\paragraph{Question 16}.\\
\begin{lstlisting}
SOL_1 (x, y, T) :
    u <- ""
    v <- ""
    n <- x.taille
    m <- y.taille
    i <- n
    j <- m

    si (n==0) alors :
        u <- m*"-"
        v <- y
        return (u,v)
    sinon si (m==0) alors :
        u <- x
        v <- n*"-"
        return (u,v)

    tant que (i >= 1) or (j >= 1) faire :

        si (i == 1) alors :
            u <- x[i-1] + u
            si (j==1) faire:
                v <- y[j-1] + v
                return (u,v)
            sinon :
                tant que (j>=1) faire :
                    u <- "-" + u
                    v <- y[j-1] + v
                    j <- j -1
                v <- y[j-1] + v
                retourne (u, v)

        sinon si j == 1 alors :
            v <- y[i-1] + v
            tant que (i>=1) alors :
                v <- "-" + v
                u <- x[i-1] + u
                i <- i -1
            u <- u[i-1] + u
            retourne (u, v)

        sinon si (T[i][j]) == (T[i-1][j] + 2) alors :
            u <- x[i-1] + u
            v <- "-" + v
            i <- i - 1

        sinon si (T[i][j]) == (T[i][j-1] + 2) alors :
            u <- "-" + u
            v <- y[j-1] + v
            j <- j - 1

        sinon si T[i][j] == (T[i-1][j-1] + cout.c_sub (x[i-1], yfaire [j-1])) :
            u <- x[i-1] + u
            v <- y[j-1] + v
            i <- i - 1
            j <- j - 1
    retourne (u, v)
\end{lstlisting}
\paragraph{Question 17}.\\
\paragraph{Question 18}.\\
\paragraph{Tâche B}.\\

\newpage
\section{Amélioration de la compléxité spatiale du calcul de la distance}
\paragraph{Question 19}
\paragraph{Question 20}
\paragraph{Tâche C}

\newpage
\section{Amélioration de la compléxité spatiale du calcul d'un alignement optimal par la méthode "diviser pour régner"}

\paragraph{Question 21}
Pseudo-code de la fonction mot\_gaps :
\begin{lstlisting}
mot_gaps(k):
    res <- ""
    pour i allant de 0 a k-1 faire:
        res <- res + "-"
    retourne res
\end{lstlisting}
\paragraph{Question 22}
Pseudo-code de la fonction align\_lettre\_mot :
\begin{lstlisting}
align_lettre_mot(lettre, mot):
    i <- 0
    tant que lettre != mot[i] et i < mot.taille:
        i <- i + 1
    si i < mot.taille alors:
        x <- mot_gaps(i) + lettre + mot_gaps(mot.taille - i - 1)
    sinon:
        mot <- mot + "-"
        lettre <- mot_gaps(mot.taille) + lettre
    retourne(lettre,mot)

\end{lstlisting}

\paragraph{Question 23}
\paragraph{Question 24}
Pseudo-code de la fonction SOL\_2 en considérant que l'on possède une fonction coupure
\begin{lstlisting}
SOL_2(x, y):
    si y.taille == 0 alors:
        retourne (x,mot_gaps(x.taille))
    si x.taille == 1 et si y.taille == 1 alors:
        si x == y alors :
            retourne (x,y)
        sinon retourne (x+"-", "-"+y)
    sinon:
        res1 <- SOL_2(x[:x.taille/2],y[:coupure(x,y)])
        res2 <- SOL_2(x[x.taille/2:],y[coupure(x,y:)])
        retourne (res1[0]+res2[0], res1[1]+res2[1])
\end{lstlisting}
\paragraph{Question 25}
Pseudo-code de la fonction coupure :
\begin{lstlisting}
coupure (x, y, T) :
    n <- len(x)+1
    m <- len(y)+1
    p <- (n-1)//2
    I <- np.zeros((n, m))
    i <- 1
    j <- 1

    pour i allant de 1 a n faire:
        pour j allant de 1 a  m faire :

            si (i<p) faire :
                I[i][j] <- 0

            sinon si (i==p) faire :
                I[i][j] <- j

            sinon :
                s <- min (T[i-1][j], T[i][j-1], T[i-1][j-1])

                si (s == T[i-1][j]) faire :
                    I[i][j] <- I[i-1][j]

                sinon si (s == T[i][j-1]) faire :
                    I[i][j] <- I[i][j-1]

                sinon si (s == T[i-1][j-1]) faire :
                    I[i][j] <- I[i-1][j-1]
    return (I[n-1][m-1])
\end{lstlisting}
\paragraph{Question 26}
\paragraph{Question 27}
\paragraph{Question 28}
\paragraph{Tâche D}
\paragraph{Question 29}


\end{document}
