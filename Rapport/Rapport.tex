\documentclass[12pt]{article}
\input{/Users/ClementMalonda/Documents/Licence/Fiche/lib/General}
\usepackage{listings}

\begin{document}
\title{Mini-Projet : Alignement de séquences}
%\title{LU3IN003 - Sorbonne Université}
\author{Amann Emmanuelle \& Malonda Clément}
\date{}
\maketitle
\tableofcontents
\newpage

\section{Introduction}
\paragraph{Question 1}.\\
Soient $(\bar x,\bar y)$ et $(\bar u, \bar v)$ deux alignements respectivement de $(x, y)$ et $(u, v)$. $\bar x$ et $\bar y$ sont aligné et donc sont de même longueur $(|\bar x| = |\bar y|)$. De même pour l'alignement $(\bar u, \bar v)$, $(|\bar x| = |\bar y|)$.\\
A partir de ces deux affirmations, nous pouvont dire que la concaténation de $\bar x$ et $\bar u$, $\bar x.\bar u$ ainsi que la concaténation de $\bar y$ et $\bar v$, $\bar y.\bar v$ sont de même longueur.

\paragraph{Question 2}.\\
Soient $x \in \sum^*$ un mot de longueur $n$ et $y \in \sum^*$ un mot de longueur $m$, la longueur maximale d'un alignement de $(x,y)$ est $n+m-1$.\\
On prend par exemple $x=ATCG$ et $y=GCTGA$, l'aligement de longueur maximale est :

$$\bar x : ATCG\_\_\_\_$$
$$\bar y : \_\_\_GCTGA$$

\section{Méthode naïve par énumération}
\paragraph{Question 3}
\paragraph{Question 4}
\paragraph{Question 5}
\paragraph{Question 6}
\paragraph{Tâche A}
Il est possible de résoudre les instances fournies en moins d'une minute pour celles qui sont de tailles 10 et 12.\\
La consommation mémoire nécessaire au fonctionnement de cette méthode est d'environ 4900Ko.


\section{Programmation dynamique}
\subsection{pour le calcul de la distance d'édtion}
%\input{Réponse partie 2.1}
\subsection{pour le calcul d'un alignement optimal}
%\input{Réponse partie 2.2}
\section{Amélioration de la compléxité spatiale du calcul de la distance}
%\input{Réponse partie 3}
\section{Amélioration de la compléxité spatiale du calcul d'un alignement optimal par la méthode "diviser pour régner"}
%\input{Réponse partie 4}

\end{document}
